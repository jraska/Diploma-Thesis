% !TEX TS-program = pdflatex
% !TEX encoding = UTF-8 Unicode
%\documentclass{article}

\documentclass[ing,male,java,dept460,oneside]{diploma}

\usepackage[czech]{babel}
\usepackage[T1]{fontenc} 
\usepackage[utf8]{inputenc}
\usepackage{color}
\usepackage{geometry}
\usepackage{float}
\usepackage{graphicx}
\usepackage{amsmath}
%\usepackage{amsthm}
\usepackage[stable]{footmisc}
\usepackage{hyperref}
\usepackage{amsfonts}
 \usepackage{bbm} 
 \usepackage{booktabs}
 \usepackage{url}
 
 
 
 \ThesisAuthor{Josef Raška}

% U bakalarske praxe neni nutne nazev zadavat
\ThesisTitle{Tos ten android ze}

% U bakalarske prace neni nutne anglicky nazev zadavat
\EnglishThesisTitle{The Android stuff}

\SubmissionDate{29. dubna 2016}

\PrintPublicationAgreement{true}


\Thanks{Podekovani \newline Dalsi lajna }


\CzechAbstract{Cesky abstrakt}

\CzechKeywords{vlastní číslo, vlastní vektor, vlastní dvojice, aplikace vlastních čísel, mocninná metoda, Lanczosova metoda, předpodmínění}

\EnglishAbstract{English abstract}

\EnglishKeywords{Android, development}
 
\title{Bakalářská práce}
\author{Josef Raška \(ras0029\)}
\numberwithin{equation}{section}
\newtheorem{priklad}{Příklad}[section]
\newtheorem{veta}{Věta}[section]
\newtheorem{alg}{Algoritmus}[section]
\begin{document}
%\maketitle
\MakeTitlePages
\urlstyle{same}

\tableofcontents
\listoffigures
\listoftables
\lstlistoflistings

\newpage

\section{Úvod}

\section{Základní definice }

\section{Motivace}
Bo chcu


\subsection{Podnadpis\footnote{Inspirace v \cite{saad}}}
VText atd
\newline
 (\url{http://www.youtube.com/watch?v=j-zczJXSxnw})
\subsubsection{Vlastní frekvence}
Proste je sva!


\section{Závěr}
Cílem práce atd.


\begin{thebibliography}{99}
\small
	\bibitem{saad}{SAAD, Y. \textit{Numerical methods for large eigenvalue problems}. Rev. ed. Philadelphia: Society for Industrial and Applied Mathematics, c2011, xvi, 276 p. ISBN 978-161-1970-722. \newline Dostupné z: \url{http://www-users.cs.umn.edu/~saad/eig_book_2ndEd.pdf}}
	\bibitem{rektorys}{REKTORYS, K. ČVUT. \textit{Matematika V. Obyčejné parciální a diferenciální rovnice}. Praha: ČVUT, 1989.}
	\bibitem{nummetody}{VONDRÁK, V.,POSPÍŠIL L. VŠB - TUO. \textit{Numerické metody 1}. Ostrava: Matematika pro inženýry 21. století, 2011. \newline Dostupné z: \url{http://mi21.vsb.cz/sites/mi21.vsb.cz/files/unit/numericke_metody.pdf}}
  \end{thebibliography}

  \appendix

  \section{Zdrojové kódy}
  Kódy lze nalézt i na přiloženém CD.
  \subsection{Základní implementace vybraných metod}


\end{document}